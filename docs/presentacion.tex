% Preamble
\documentclass{beamer}
\usepackage[spanish]{babel}

% Packages
\usepackage{amsmath}
\usepackage[utf8]{inputenc}
\usepackage[T1]{fontenc}
\usepackage{graphicx}
\usepackage{algorithmicx}
\usepackage{algpseudocode}
\usepackage{caption}
\usepackage{courier}

\DeclareMathOperator{\atantwo}{atan2}

\captionsetup{justification=centering, font={scriptsize}, skip=0pt}

% Set bold vectors to satisfy requirements
\renewcommand\vec[1]{\ifstrequal{#1}{0}{\ensuremath{\mathbf{0}}}{\ensuremath{\boldsymbol{#1}}}}

\usetheme[compress]{Berlin}
\usecolortheme{wolverine}
\setbeamertemplate{page number in head/foot}[framenumber]
\setbeamercolor{institute in head/foot}{parent=palette primary}

\title[Dinámica molecular regida por el paso temporal]{Dinámica molecular regida por el paso temporal}
\subtitle{72.25 - Simulación de Sistemas}
\author[Flores Lucey, Llanos]{Alejo Flores Lucey\inst{1} \and Nehuén Gabriel Llanos\inst{2}}
\institute[Instituto Tecnológico de Buenos Aires]
{
    \inst{1}
    \href{mailto:afloreslucey@itba.edu.ar}{afloreslucey@itba.edu.ar}\\
    Legajo 62622
    \and
    \inst{2}
    \href{mailto:nllanos@itba.edu.ar}{nllanos@itba.edu.ar}\\
    Legajo 62511
}
\date{2024 1C | Grupo Nº3}
\titlegraphic{\includegraphics[height=0.5cm]{./itba}}

\makeatletter
\beamer@theme@subsectionfalse
\makeatother

\AtBeginSection[]{
    \begin{frame}
        \begin{beamercolorbox}[sep=8pt,center]{title}
            \usebeamerfont{title}\insertsection
        \end{beamercolorbox}
    \end{frame}
}

\begin{document}

    \begin{frame}
        \titlepage
    \end{frame}

    \section{Oscilador Puntual Amortiguado}

        \begin{frame}{Oscilador Puntual Amortiguado}
            \begin{itemize}
                \item Verlet Original
                \begin{equation*}
                    \begin{cases}
                        r(t + \Delta t) = 2 r(t) - r(t - \Delta t) + \Delta t^2 a(r(t), v(t))\\
                        r(t + \Delta t) = f(r(t), r(t - \Delta t), v(t))
                    \end{cases}
                \end{equation*}
                \begin{equation*}
                      \begin{cases}
                          v(t) = \frac{r(t + \Delta t) - r(t - \Delta t)}{2 \Delta t}\\
                          v(t) = g(r(t + \Delta t), r(t - \Delta t))
                      \end{cases}
                \end{equation*}
                \item Beeman
                \item Gear Predictor-Corrector
                \item Verlet 2.0 (se obtiene $r$ y $v$ para el mismo $\Delta t$.)
                \begin{equation*}
                    \begin{cases}
                        r(t + \Delta t) = f(r(t), r(t - \Delta t), v(t))\\
                        r(t + 2 \Delta t) = f(r(t + \Delta t), r(t), v(t))\\
                        v(t + \Delta t) = g(r(t + 2 \Delta t), r(t))
                    \end{cases}
                \end{equation*}
            \end{itemize}
        \end{frame}

        \begin{frame}{Oscilador Puntual Amortiguado: Resultados}
            \begin{figure}[H!]
                \includegraphics[width=0.9\textwidth]{./oscilador_resultados}
                \label{fig:oscilador_1}
            \end{figure}
        \end{frame}

        \begin{frame}{Oscilador Puntual Amortiguado: Error vs $\Delta t$}
            \begin{figure}[H!]
                \includegraphics[width=0.9\textwidth]{./oscilador_error}
                \label{fig:oscilador_2}
            \end{figure}
        \end{frame}

    % HASTA ACA ESTÁ CAMBIADO

    \section{Introducción}

        \begin{frame}{Introducción}
            \begin{itemize}
                \item \textbf{Objetivo:}
                \begin{itemize}
                    \item Simular un sistema de partículas en un espacio bidimensional.
                    \item Predecir propiedades de sistemas físicos a nivel de partículas.
                    \item Relacionar observables macroscópicos con dinámicas microscópicas.
                \end{itemize}
                \item \textbf{Modo de simulación:}
                \begin{itemize}
                    \item Simulación dirigida por eventos.
                    \item Se calculan los tiempos de colisión.
                    \item Se avanzan los estados hasta ese tiempo.
                \end{itemize}
            \end{itemize}
        \end{frame}

        \subsection{Sistema Real}

            \begin{frame}{Sistema Real}
                \begin{itemize}
                    \item Las partículas tienen un radio $r$ y una velocidad inicial $\vec{v}$.
                    \item Se estudia como colisionan entre sí y con las paredes del contenedor.
                \end{itemize}
                \begin{minipage}[t]{0.5\textwidth}
                    \begin{itemize}
                        \item \textbf{Aplicaciones:}
                        \begin{itemize}
                            \item Movimiento de moléculas de gas.
                            \item Dinámicas de reacciones químicas.
                            \item Desarrollo de videojuegos.
                        \end{itemize}
                    \end{itemize}
                \end{minipage}
                \hfill
                % HASTA ACA ESTA CAMBIADO
                \begin{minipage}[t]{0.45\textwidth}
                    \begin{figure}[H]
                        \centering
                        \includegraphics[width=\linewidth]{./angrybirds}
                        \label{fig:angry_birds}
                    \end{figure}
                \end{minipage}
            \end{frame}

        \subsection{Fundamentos}

            \begin{frame}{Fundamentos}
                \begin{itemize}
                    \item Las partículas son una 5-upla $(x, y, R, v_x, v_y)$
                    \item Las partículas se mueven en linea recta hasta que ocurre una colisión:
                        \begin{equation*}
                            \vec{x_i}(t+1) = \vec{x_i}(t) + \vec{v_i}(t) t_c\ ; \ t_c = \text{tiempo mínimo de colisión}
                        \end{equation*}
                    \item Se busca el tiempo en el que las partículas colisionan:
                        \begin{equation*}
                            \left( x_i - x_j \right) ^2 + \left( y_i - y_j \right) ^2 = \left( R_i + R_j \right) ^2
                        \end{equation*}
                    \item Las colisiones son elásticas y se conserva el impulso.
                \end{itemize}
            \end{frame}

    \section{Implementación}

        \subsection{Arquitectura}

            \begin{frame}{Diagrama UML}
                \begin{figure}[htbp]
                    \centering
                    \includegraphics[width=\textwidth]{./architecture}
                    \label{fig:architecture}
                \end{figure}
            \end{frame}

        \subsection{Algoritmo}

            \begin{frame}{Pseudocódigo del algoritmo implementado}{}
                \begin{algorithmic}[1]
                    \ttfamily \scriptsize
                    \State Create output file
                    \State collisions $\gets$ new OrderedSet()
                    \ForAll{p $\in$ particles}
                        \State collisions $\gets$ \Call{CalculateCollisionsWithWalls}{p}
                        \ForAll{q $\in$ particles}
                            \State collisions $\gets$ \Call{CalculateCollisionsBetweenParticles}{p, q}
                        \EndFor
                    \EndFor
                    \While{$i < eventCount$}
                        \State first $\gets$ \Call{PopFirstCollision}{collisions}
                        \ForAll{p $\in$ particles}
                            \State \Call{Move}{p, first.time}
                        \EndFor
                        \State \Call{UpdateCollisionsTime}{first.time}
                        \State \Call{WriteOutput}{particles}
                        \State \Call{UpdateVelocities}{first.p1, first.p2}
                        \State \Call{CalculateNewCollisions}{first}
                        \State $i \gets i + 1$
                    \EndWhile
                \end{algorithmic}
            \end{frame}

    \section{Simulaciones}

        \subsection{Parámetros de entrada}

            \begin{frame}{Parámetros de entrada}
                \begin{itemize}
                    \item Parámetros de entrada fijos:
                    \begin{itemize}
                        \item Número de partículas ($N$): \alert{$300$}
                        \item Cantidad de eventos ($E$): \alert{$20\,000$}
                        \item Longitud del plano ($L$): \alert{$0.1 m$}
                        \item Radio de partículas ($r$): \alert{$0.001 m$}
                        \item Radio del obstáculo ($R$): \alert{$0.005 m$}
                        \item Masa de partículas ($m$): \alert{$1 kg$}
                        \item Masa del obstáculo ($m$): \alert{$3 kg$}
                    \end{itemize}
                    \item Parámetros de entrada variables:
                    \begin{itemize}
                        \item Módulo de la velocidad inicial ($v_0$): \alert{$\{1, 3, 6, 10\} m/s$}
                    \end{itemize}
                \end{itemize}
            \end{frame}

        \subsection{Observables}

            \begin{frame}{Presión en paredes y obstáculo en función del tiempo ($P(t)$)}
                \begin{itemize}
                    \item Presión en paredes en función del tiempo:
                    \begin{equation*}
                        P_{Paredes}(t) = \frac{\sum \Delta V_{n\ (Paredes)}}{\Delta t \cdot L \cdot 4}
                    \end{equation*}
                    \begin{equation*}
                        v_{n\ (Paredes\ Sup.\ y\ Inf.)} = \left| v \cdot \sin(\alpha) \cdot 2 \right|
                    \end{equation*}
                    \begin{equation*}
                        v_{n\ (Paredes\ Izq.\ y\ Der.)} = \left| v \cdot \cos(\alpha) \cdot 2 \right|
                    \end{equation*}
                    \item Presión en el obstáculo en función del tiempo:
                    \begin{equation*}
                        P_{Obst\acute{a}culo}(t) = \frac{\sum \vec{v_{n\ (Obst\acute{a}culo)}}}{\Delta t \cdot 2\pi \cdot R}
                    \end{equation*}
                    \begin{equation*}
                          \vec{v_{n\ (Obst\acute{a}culo)}} = \left| {v \cdot \cos \left( \atantwo \left(\frac{L}{2} - y, \frac{L}{2} - x \right) \right) \cdot 2} \right|
                    \end{equation*}
                \end{itemize}
            \end{frame}

            \begin{frame}{Tiempo en el que el nro. de choques alcanza el 20\% de $N$ }
                \begin{itemize}
                    \item Se realizan diez (10) corridas.
                    \item Se estudia el tiempo en el que el 20\% de las partículas colisionan por primera vez con el obstáculo.
                    \begin{equation*}
                        t_{\text{prom}} = \frac{\sum_{i=1}^{10} t_i}{10} \ ,\ t_i\text{: tiempo en corrida }i
                    \end{equation*}
                    \begin{equation*}
                        \sigma = \sqrt{\frac{1}{9} \sum_{i=1}^{10} (t_i - t_{\text{prom}})^2}
                    \end{equation*}
                \end{itemize}
            \end{frame}

            \begin{frame}{Número de choques por unidad de tiempo}
                \begin{itemize}
                    \item Se calcula la pendiente de la curva mediante el método de regresión lineal simple
                    \begin{equation*}
                        \begin{split}
                            m_k(\iota) = \frac{\iota \sum_{i=1}^\iota x_i y_i - \sum_{i=1}^\iota x_i \sum_{i=1}^\iota y_i }{\iota \sum_{i=1}^\iota x_i^2 - \left(\sum_{i=1}^\iota x_i \right)^2}
                            \ , 1 \leq k \leq 10
                            \\ \iota = \text{cantidad de choques en el intervalo de tiempo}
                        \end{split}
                    \end{equation*}
                    \item Se promedian los valores obtenidos de cada una de las diez (10) corridas y se calcula su error asociado:
                    \begin{equation*}
                            m_{prom} = \frac{1}{10} \sum_{i=1}^{10} m_i
                        \text{ ; }
                            \sigma = \sqrt{\frac{1}{9} \sum_{i=1}^{10} (m_i - m_{\text{prom}})^2}
                    \end{equation*}
                \end{itemize}
            \end{frame}

            \begin{frame}{Desplazamiento cuadrático medio ($DCM$)}
                \begin{itemize}
                    \item Se realiza para cuatro (4) velocidades y diez (10) corridas cada una.
                    \item Se divide el intervalo de tiempo en cien (100) $\Delta t$.
                    \item Se obtiene el desplazamiento cuadrático medio para cada $\Delta t$:
                    \begin{equation*}
                        DCM_{\Delta t} = \frac{\sum_{i=1}^{10} \left( x_{t_i} - x_0 \right)^2 + \left( y_{t_i} - y_0 \right)^2}{10}
                    \end{equation*}
                \end{itemize}
            \end{frame}

            \begin{frame}{Coeficiente de difusión ($D$)}
                \begin{itemize}
                    \item Se obtiene la pendiente de la curva de $DCM$ en función del tiempo mediante el método
                    de regresión lineal simple para cada velocidad.
                    \begin{equation*}
                        m_v = \frac{100 \sum_{i=1}^{100} x_i y_i - \sum_{i=1}^{100} x_i \sum_{i=1}^{100} y_i }{100 \sum_{i=1}^{100} x_i^2
                        - \left(\sum_{i=1}^{100} x_i \right)^2},\ v \in \{1, 3, 6, 10\}
                    \end{equation*}
                    \item Se obtiene el coeficiente de difusión:
                    \begin{equation*}
                        D_v = \frac{m_v}{4},\ v \in \{1, 3, 6, 10\}
                    \end{equation*}
                \end{itemize}
            \end{frame}

    \section{Resultados}

        \subsection{Animación del sistema}

            \begin{frame}{Animación del sistema}{}
                \vspace*{-0.3cm}
                \begin{minipage}[t]{0.49\textwidth}
                    \begin{figure}[H!]
                        \includegraphics[width=\textwidth]{./animacion_marte_1}
                        \caption*{Véase la animación completa en \url{https://youtu.be/1PgQa7WLQ6g}.}
                        \label{fig:marte_1}
                    \end{figure}
                    \vspace*{-0.5cm}
                    \begin{beamercolorbox}[sep=5pt,center]{block body}
                        \centering
                        \small{Día de partida: 0}
                    \end{beamercolorbox}
                \end{minipage}
                \hfill
                \begin{minipage}[t]{0.49\textwidth}
                    \begin{figure}[H!]
                        \includegraphics[width=\textwidth]{./animacion_marte_2}
                        \caption*{Véase la animación completa en \url{https://youtu.be/MAS-4i-Jz9Q}.}
                        \label{fig:marte_2}
                    \end{figure}
                    \vspace*{-0.5cm}
                    \begin{beamercolorbox}[sep=5pt,center]{block body}
                        \centering
                        \small{Día de partida: 172}
                    \end{beamercolorbox}
                \end{minipage}
            \end{frame}

        \subsection{Elección del $\Delta t$}

            \begin{frame}{Energía perdida en función del tiempo}{Elección del $\Delta t$}
                    \begin{figure}[H!]
                        \includegraphics[width=0.9\textwidth]{./energia_perdida_vs_tiempo}
                        \label{fig:marte_3}
                    \end{figure}
                    \begin{beamercolorbox}[sep=5pt,center]{block body}
                        \centering
                        \small{Estudio en 50 días}
                    \end{beamercolorbox}
            \end{frame}

            \begin{frame}{Promedio de energía perdida vs $\Delta t$}{Elección del $\Delta t$}
                \begin{figure}[H!]
                    \includegraphics[width=0.9\textwidth]{./promedio_energia_perdida_vs_dt}
                    \label{fig:marte_4}
                \end{figure}
                \begin{beamercolorbox}[sep=5pt,center]{block body}
                    \centering
                    \small{Estudio en 50 días}
                \end{beamercolorbox}
            \end{frame}

        \subsection{Momento óptimo de partida para arribar a Marte}

            \begin{frame}{Distancia mínima a Marte vs día de partida}{Momento óptimo de partida para arribar a Marte}
                \begin{figure}[H!]
                    \includegraphics[width=0.9\textwidth]{./distancia_a_marte_vs_dia_de_partida}
                    \label{fig:marte_5}
                \end{figure}
                \begin{beamercolorbox}[sep=5pt,center]{block body}
                    \centering
                    \small{$\Delta t = 60s$ ; Misión exitosa si $d_{nave-marte} < 3520 km$}
                \end{beamercolorbox}
            \end{frame}

            \begin{frame}{Distancia mínima a Marte vs hora de partida}{Momento óptimo de partida para arribar a Marte. A partir de 171d}
                \begin{figure}[H!]
                    \includegraphics[width=0.9\textwidth]{./distancia_a_marte_vs_hora_de_partida}
                    \label{fig:marte_6}
                \end{figure}
                \begin{beamercolorbox}[sep=5pt,center]{block body}
                    \centering
                    \small{$\Delta t = 60s$ ; Misión exitosa si $d_{nave-marte} < 3520 km$}
                \end{beamercolorbox}
            \end{frame}

            \begin{frame}{Distancia mínima a Marte vs minuto de partida}{Momento óptimo de partida para arribar a Marte. A partir de 171d 22h}
                \begin{figure}[H!]
                    \includegraphics[width=0.9\textwidth]{./distancia_a_marte_vs_minuto_de_partida}
                    \label{fig:marte_7}
                \end{figure}
                \begin{beamercolorbox}[sep=5pt,center]{block body}
                    \centering
                    \small{$\Delta t = 60s$ ; $d_{nave-marte} = 2547.98 km$}
                \end{beamercolorbox}
            \end{frame}

        \subsection{Velocidad de la nave vs tiempo}

            \begin{frame}{Módulo de la velocidad de la nave vs tiempo}{Partida de la nave: 171d 23h 17m}
                \begin{figure}[H!]
                    \includegraphics[width=0.9\textwidth]{./velocity_vs_time_for_travel_to_mars}
                    \label{fig:marte_8}
                \end{figure}
                \begin{beamercolorbox}[sep=5pt,center]{block body}
                    \centering
                    \small{$\Delta t = 60s$}
                \end{beamercolorbox}
            \end{frame}

        \subsection{Variación de la velocidad incial de la nave}

            \begin{frame}{Distancia mínima a Marte vs velocidad inicial de la nave}{Partida de la nave: 171d 23h 17m}
                \begin{figure}[H!]
                    \includegraphics[width=0.9\textwidth]{./min_distance_vs_v0_logaritmica_line_in_10^4_scatter}
                    \label{fig:marte_9}
                \end{figure}
                \begin{beamercolorbox}[sep=5pt,center]{block body}
                    \centering
                    \small{$\Delta t = 60s$ ; Misión exitosa si $d_{nave-marte} < 10^4 km$}
                \end{beamercolorbox}
            \end{frame}

            \begin{frame}{Distancia mínima a Marte vs velocidad inicial de la nave}{Partida de la nave: 171d 23h 17m}
                \begin{figure}[H!]
                    \includegraphics[width=0.9\textwidth]{./min_distance_vs_v0_logaritmica_line_in_10^4_reduced_scatter}
                    \label{fig:marte_10}
                \end{figure}
                \begin{beamercolorbox}[sep=5pt,center]{block body}
                    \centering
                    \small{$\Delta t = 60s$ ; Misión exitosa si $d_{nave-marte} < 10^4 km$}
                \end{beamercolorbox}
            \end{frame}

            \begin{frame}{Tiempo de vuelo vs velocidad inicial de la nave}{Partida de la nave: 171d 23h 17m}
                \begin{figure}[H!]
                    \includegraphics[width=0.9\textwidth]{./time_vs_v0_for_v_less_10^4_scatter}
                    \label{fig:marte_11}
                \end{figure}
                \begin{beamercolorbox}[sep=5pt,center]{block body}
                    \centering
                    \small{$\Delta t = 60s$ ; Misión exitosa si $d_{nave-marte} < 10^4 km$}
                \end{beamercolorbox}
            \end{frame}

        \subsection{Sistema con Júpiter}

            \begin{frame}{Animación del sistema}{}
                \vspace*{-0.3cm}
                \begin{minipage}[t]{0.49\textwidth}
                    \begin{figure}[H!]
                        \includegraphics[width=\textwidth]{./animacion_jupiter_1}
                        \caption*{Véase la animación completa en \url{https://youtu.be/uGcs41rDAmk}.}
                        \label{fig:jupiter_1}
                    \end{figure}
                    \vspace*{-0.5cm}
                    \begin{beamercolorbox}[sep=5pt,center]{block body}
                        \centering
                        \small{Día de partida: 0}
                    \end{beamercolorbox}
                \end{minipage}
                \hfill
                \begin{minipage}[t]{0.49\textwidth}
                    \begin{figure}[H!]
                        \includegraphics[width=\textwidth]{./animacion_jupiter_2}
                        \caption*{Véase la animación completa en \url{https://youtu.be/GijCuivfah8}.}
                        \label{fig:jupiter_2}
                    \end{figure}
                    \vspace*{-0.5cm}
                    \begin{beamercolorbox}[sep=5pt,center]{block body}
                        \centering
                        \small{Día de partida: 896}
                    \end{beamercolorbox}
                \end{minipage}
            \end{frame}

            \begin{frame}{Distancia mínima a Júpiter vs día de partida}{Momento óptimo de partida para arribar a Júpiter}
                \begin{figure}[H!]
                    \includegraphics[width=0.9\textwidth]{./distancia_a_jupiter_vs_dia_de_partida}
                    \label{fig:jupiter_3}
                \end{figure}
                \begin{beamercolorbox}[sep=5pt,center]{block body}
                    \centering
                    \small{$\Delta t = 60s$ ; Misión exitosa si $d_{nave-jupiter} < 71000 km$}
                \end{beamercolorbox}
            \end{frame}

            \begin{frame}{Distancia mínima a Júpiter vs hora de partida}{Momento óptimo de partida para arribar a Júpiter. A partir de 895d}
                \begin{figure}[H!]
                    \includegraphics[width=0.9\textwidth]{./distancia_a_jupiter_vs_hora_de_partida}
                    \label{fig:jupiter_4}
                \end{figure}
                \begin{beamercolorbox}[sep=5pt,center]{block body}
                    \centering
                    \small{$\Delta t = 60s$ ; Misión exitosa si $d_{nave-jupiter} < 71000 km$}
                \end{beamercolorbox}
            \end{frame}

            \begin{frame}{Distancia mínima a Júpiter vs minuto de partida}{Momento óptimo de partida para arribar a Júpiter. A partir de 896d 1h}
                \begin{figure}[H!]
                    \includegraphics[width=0.9\textwidth]{./distancia_a_jupiter_vs_minuto_de_partida}
                    \label{fig:jupiter_5}
                \end{figure}
                \begin{beamercolorbox}[sep=5pt,center]{block body}
                    \centering
                    \small{$\Delta t = 60s$ ; $d_{nave-jupiter} = 705125 km$}
                \end{beamercolorbox}
            \end{frame}

            \begin{frame}{Distancia mínima a Júpiter vs velocidad inicial de la nave}{Partida de la nave: 896d 1h 52m}
                \begin{figure}[H!]
                    \includegraphics[width=0.9\textwidth]{./min_distance_vs_v0_logarithmic_jupiter}
                    \label{fig:jupiter_6}
                \end{figure}
                \begin{beamercolorbox}[sep=5pt,center]{block body}
                    \centering
                    \small{$\Delta t = 60s$ ; Misión exitosa si $d_{nave-jupiter} < 10^6 km$}
                \end{beamercolorbox}
            \end{frame}

            \begin{frame}{Tiempo de vuelo vs velocidad inicial de la nave}{Partida de la nave: 896d 1h 52m}
                \begin{figure}[H!]
                    \includegraphics[width=0.9\textwidth]{./travel_time_vs_v0_logarithmic_jupiter}
                    \label{fig:jupiter_7}
                \end{figure}
                \begin{beamercolorbox}[sep=5pt,center]{block body}
                    \centering
                    \small{$\Delta t = 60s$ ; Misión exitosa si $d_{nave-jupiter} < 10^6 km$}
                \end{beamercolorbox}
            \end{frame}

    \section{Conclusiones}

        \begin{frame}{Conclusiones}
            \begin{itemize}
                \item El óptimo $\Delta t$ tiene un mínimo.
                Llega un punto que si se sigue disminuyendo, el error es mayor.
                \item Los cambios en la velocidad inicial de la nave resultan en una mayor distancia mínima a Marte.
                \item A medida que la nave se aleja del sol, su aceleración disminuye.
                \item Estas observaciones se pueden extrapolar a otros sistemas planetarios.
            \end{itemize}
        \end{frame}

        \begin{frame}
            \begin{beamercolorbox}[sep=8pt,center]{title}
                \usebeamerfont{title}{¡Muchas gracias!}
            \end{beamercolorbox}
        \end{frame}

\end{document}
